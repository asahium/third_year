\documentclass[11pt]{article}

\makeatletter
\newcommand{\skipitems}[1]{%
	\addtocounter{\@enumctr}{#1}%
}
\makeatother

\newcommand{\numpy}{{\tt numpy}}
\usepackage{amsfonts}
\usepackage{amsmath}
\usepackage{graphics}
\usepackage{amsthm,amstext,amsfonts,bm,amssymb}
\usepackage{graphicx}
\graphicspath{ {./images/} }
\usepackage{indentfirst}
\setlength{\parindent}{5ex}
\setlength{\parskip}{1em}
\usepackage[utf8x]{inputenc} 
\usepackage[russian]{babel}
\topmargin -.5in
\textheight 9in
\oddsidemargin -.25in
\evensidemargin -.25in
\textwidth 7in


\begin{document}
	
	\author{Биктимиров Данила, группа 204}
	\title{ДЗ 3}
	\date{}
	\maketitle
	
	\medskip
	
	\begin{enumerate}
		
		\item 
		\item 
		\item 
		\item 
		\item 
		\item 
		\item 
		\item Машина должна входить по ленте взад-вперёд, стирая по паре символов. При этом она помнит номер стёртого символа за счёт своих состояний. Если где-то несовпадение, то остаток записи стирается, и выдаётся ответ 0. Если всё время совпадения, то в конце остаётся 0 или 1 символов. Тогда выдаём ответ 1.
		\item  Если сдвиги допускаются на ограниченную длину, то за счёт увеличения числа состояний это можно свести к сдвигам на единицу.
		\item 
		\item Множество машин Тьюринга с <=k состояниями конечно, поэтому с их помощью можно напечатать лишь конечное множество чисел. Поэтому существует наименьшее число, зависящее от k, которое напечатать таким способом нельзя. Обозначим его через f(k).
		
		Предположим, что есть алгоритм вычисления C(n). Тогда, вызывая подпрограмму вычисления на элементах n=1,2,... , мы рано или поздно получим значение функции строго больше k. Первое такое n и есть f(k). То есть f оказывается вычислимой. 
		
		если f(k) вычислима, то f(2k+2) тоже вычислима. Пусть вычисляющая это дело машина (с входом) имеет m состояний. Строим машину без входа, которая сначала на пустой ленте рисует m единиц (или m+1 -- тут всё зависит от формата ввода), потом передаёт управление предыдущей машине с уже созданными входными данными. У машины без входа будет не более 2m+2 состояний, а выдаёт она число f(2m+2), и это уже является противоречием.
		
	\end{enumerate}
\end{document}