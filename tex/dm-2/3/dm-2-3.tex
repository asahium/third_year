\documentclass[11pt]{article}

\makeatletter
\newcommand{\skipitems}[1]{%
	\addtocounter{\@enumctr}{#1}%
}
\makeatother

\newcommand{\numpy}{{\tt numpy}}
\usepackage{amsfonts}
\usepackage{amsmath}
\usepackage{graphics}
\usepackage{amsthm,amstext,amsfonts,bm,amssymb}
\usepackage{graphicx}
\graphicspath{ {./images/} }
\usepackage{indentfirst}
\setlength{\parindent}{5ex}
\setlength{\parskip}{1em}
\usepackage[utf8x]{inputenc} 
\usepackage[russian]{babel}
\topmargin -.5in
\textheight 9in
\oddsidemargin -.25in
\evensidemargin -.25in
\textwidth 7in


\begin{document}
	
	\author{Биктимиров Данила, группа 204}
	\title{ДЗ 3}
	\date{}
	\maketitle
	
	\medskip
	
	\begin{enumerate}
		
		\item 
		\item 
		\item 
		\item 
		\item 
		\item 
		\item 
		\item Машина должна входить по ленте взад-вперёд, стирая по паре символов. При этом она помнит номер стёртого символа за счёт своих состояний. Если где-то несовпадение, то остаток записи стирается, и выдаётся ответ 0. Если всё время совпадения, то в конце остаётся 0 или 1 символов. Тогда выдаём ответ 1.
		\item 
		\item 
		\item 
		
	\end{enumerate}
\end{document}