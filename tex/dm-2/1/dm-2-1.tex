\documentclass[11pt]{article}

\makeatletter
\newcommand{\skipitems}[1]{%
	\addtocounter{\@enumctr}{#1}%
}
\makeatother

\newcommand{\numpy}{{\tt numpy}}
\usepackage{amsfonts}
\usepackage{amsmath}
\usepackage{graphics}
\usepackage{amsthm,amstext,amsfonts,bm,amssymb}
\usepackage{graphicx}
\graphicspath{ {./images/} }
\usepackage{indentfirst}
\setlength{\parindent}{5ex}
\setlength{\parskip}{1em}
\usepackage[utf8x]{inputenc} 
\usepackage[russian]{babel}
\topmargin -.5in
\textheight 9in
\oddsidemargin -.25in
\evensidemargin -.25in
\textwidth 7in


\begin{document}
	
	\author{Биктимиров Данила, группа 204}
	\title{ДЗ 1}
	\date{}
	\maketitle
	
	\medskip
	
	\begin{enumerate}
		
		\item  \begin{enumerate}
			\item Непустое $A$ является множеством значений всюду определённой вычислимой функции $f$. Перебирая все пары вида $(m,n)$, рассматриваем число $x=f(m)$ и вычисляем при помощи алгоритма значение $\cos x$ с $n$ знаками после запятой. Все числа $z$, встречающиеся в записи дробной части косинуса, подаём на печать. В итоге будет напечатано множество из условия, то есть оно перечислимо.
			\item Очевидно, что $z=1$ подходит. Для остальных $z$ находим каноническое разложение. Вычисляем НОД показателей степеней. Его делители проверяем на предмет принадлежности $B$. Если хотя бы один принадлежит, то $z$ принадлежит исследуемому множеству, что даёт алгоритм.
			
	\end{enumerate}
		
		
		\item Мы знаем что существует какое-то неразрешимое множество $A\in \mathbf{N}$. Возьмем его и в качестве $B$ возьмем $\mathbf{N}$, тогда все условия раотают, ответ существуют.
		
		\item Заметим, что тогда $A\cup B$ перечислимо и $C$ перечислимо. Тогда перечислимо и $C \cap B$. Тогда перечислимо $A$ и $\overline{A}$, значит $A$ разрешимо.
		
		\item Тогда $\overline{A\cup B} $ тоже разрешимо, а это $A\cap B$. $B$ так как конечно, то разрешимо. Тогда мы можем взять неразрешимое множество $A$ и конечное число элементов $B$, что $\forall x \in B : x \notin A $. Тогда все будет удовлетворять условию так как пустое множество разрешимое множество.
		
		\item Точка $(x,y)$ принадлежит графику функции $f \Leftrightarrow y=f(x)$, а такое условие проверяется при помощи алгоритма ввиду вычислимости $f$ (с учётом того, что она всюду определена).
		
		\item Пусть $f(x)=1$ на $A\in \mathbf{N}$, $f(x)=69$ на $\overline{A}$. Для $g(x)$ наоборот. Тогда $h(x)=69$ тождественно, она вычислима. При этом $f$ и $g$ невычислимы, если $A$ неразрешимо.
		
		\item Программа вычисления имеет примерно такой вид: если $x\ge x_0$, то $f:=x+k$, где $k$ -- число значений $\mathbf{N} \setminus rng f$, $x_0$ -- наибольшее значение начиная с которого строго возрастает без исключений, а для конечного множества остальных значений аргумента, значение функции находится по известной нам таблице значений.
		
		\item Пусть число классов эквивалентности равно $k$. Тогда существует конечный набор представителей этих классов: $a(1), ... , a(k)$.
		
		На вход программы подаём два числа $m$ и $n$. Нужно определить, эквивалентны они или нет. Число $m$ эквивалентно ровно одному из представителей. Запускаем программу перечисления пар из $E$, и за конечное время дожидаемся появления пары вида $(m,a(i))$. Аналогично поступаем с числом $n$, дожидаясь появления пары $(n,a(j))$. Теперь мы знаем числа $i$, $j$. Если они равны, то $m$ и $n$ эквивалентны. Если не равны, то не эквивалентны. Это даёт разрешающий алгоритм.
		
		
		
	\end{enumerate}
\end{document}