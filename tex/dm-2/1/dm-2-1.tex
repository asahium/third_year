\documentclass[11pt]{article}

\makeatletter
\newcommand{\skipitems}[1]{%
	\addtocounter{\@enumctr}{#1}%
}
\makeatother

\newcommand{\numpy}{{\tt numpy}}
\usepackage{amsfonts}
\usepackage{amsmath}
\usepackage{graphics}
\usepackage{amsthm,amstext,amsfonts,bm,amssymb}
\usepackage{graphicx}
\graphicspath{ {./images/} }
\usepackage{indentfirst}
\setlength{\parindent}{5ex}
\setlength{\parskip}{1em}
\usepackage[utf8x]{inputenc} 
\usepackage[russian]{babel}
\topmargin -.5in
\textheight 9in
\oddsidemargin -.25in
\evensidemargin -.25in
\textwidth 7in


\begin{document}
	
	\author{Биктимиров Данила, группа 204}
	\title{ДЗ 1}
	\date{}
	\maketitle
	
	\medskip
	
	\begin{enumerate}
		
		\item 
		
		
		\item Мы знаем что существует какое-то неразрешимое множество $A\in \mathbf{N}$. Возьмем его и в качестве $B$ возьмем $\mathbf{N}$, тогда все условия раотают, ответ существуют.
		\item 
		
		\item
		
		\item Точка $(x,y)$ принадлежит графику функции $f \Leftrightarrow y=f(x)$, а такое условие проверяется при помощи алгоритма ввиду вычислимости $f$ (с учётом того, что она всюду определена).
		\item
		
		\item 
		
		
		\item Пусть число классов эквивалентности равно $k$. Тогда существует конечный набор представителей этих классов: $a(1), ... , a(k)$.
		
		На вход программы подаём два числа $m$ и $n$. Нужно определить, эквивалентны они или нет. Число $m$ эквивалентно ровно одному из представителей. Запускаем программу перечисления пар из $E$, и за конечное время дожидаемся появления пары вида $(m,a(i))$. Аналогично поступаем с числом $n$, дожидаясь появления пары $(n,a(j))$. Теперь мы знаем числа $i$, $j$. Если они равны, то $m$ и $n$ эквивалентны. Если не равны, то не эквивалентны. Это даёт разрешающий алгоритм.
		
		
		
	\end{enumerate}
\end{document}