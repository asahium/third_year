\documentclass[11pt]{article}

\makeatletter
\newcommand{\skipitems}[1]{%
	\addtocounter{\@enumctr}{#1}%
}
\makeatother

\newcommand{\numpy}{{\tt numpy}}
\usepackage{amsfonts}
\usepackage{amsmath}
\usepackage{graphics}
\usepackage{amsthm,amstext,amsfonts,bm,amssymb}
\usepackage{graphicx}
\graphicspath{ {./images/} }
\usepackage{indentfirst}
\setlength{\parindent}{5ex}
\setlength{\parskip}{1em}
\usepackage[utf8x]{inputenc} 
\usepackage[russian]{babel}
\topmargin -.5in
\textheight 9in
\oddsidemargin -.25in
\evensidemargin -.25in
\textwidth 7in


\begin{document}
	
	\author{Биктимиров Данила, группа 204}
	\title{ДЗ 4}
	\date{}
	\maketitle
	
	\medskip
	
	\begin{enumerate}
		
		\item Запишем предложение $(Ez)(Q(x,z) \& not(Q(y,z)))$. Это некоторый предикат $P(x,y)$, выразимый в языке данной сигнатуры. Если $P(x,y)$ истинно в какой-то интерпретации, то понятно, что элементы x и y не равны.
		
		Напишем теперь формулу с кванторной приставкой из n кванторов существования по переменным $x(i)$, $1\le i\le n$, и далее выпишем конъюнкцию из $n(n-1)/2$ формул вида $P(x(i),x(j))$ по всем $i < j$. Ясно, что любая модель будет иметь не менее $n$ элементов, так как все $x(i)$ в ней интерпретируются по-разному.
		
		Выполнимость легко следует из существования бесконечной модели. Например, $N$, где предикат $Q$ интерпретируется как равенство. Тогда для не равных $x$, $y$ значение $P(x,y)$ будет истинно (поскольку существует $z$, равный $x$).
		
		\item Наложим условия, задающее биекцию множества на себя. Под $Q(x,y)$ тогда будет пониматься, что $x$ переходит в $y$ при такой биекции.
		
		1) $(Ax)(Ey)Q(x,y)$ -- у любого элемента есть образ
		
		2) $(Ax)(Ay)(Az)(Q(x,y)\& Q(x,z)\to y=z)$ -- условие однозначности соответствия
		
		3) $(Ax)(Ay)(Az)(Q(x,z)\& Q(y,z)\to x=y)$ -- условие инъективности
		
		4) $(Ay)(Ex)Q(x,y)$ -- условие сюръективности
		
		Теперь потребуем существование двух различных элементов a,b, которые переходят при отображении друг в друга. Для остальных элементов потребуем существование циклов длины 3 типа $x\to y\to z\to x$, где все элементы попарно различны. Это должно выполняться для всех $x$ отличных от выделенных элементов $a$,$b$.
		
		При таких условиях все конечные модели, представленные графом, будут состоять из одного цикла длиной 2 и нескольких циклов длиной 3, причём эти циклы не пересекаются по причине биективности. Добавляем ещё один пункт, а в конце берём конъюнкцию всех пяти формул.
		
		5) $(Ea)(Eb)(not(a=b) \& Q(a,b) \& Q(b,a) \& (Ax)(not(x=a)\& not(x=b) \to \\ (Ey)(Ez)(Q(x,y)\& Q(y,z)\& Q(z,x)\& not(x=y)\& not(y=z)\& not(z=x)))))$
		
		\item Берём двуместный предикат P. Пишем формулы, выражающие свойство, что P есть отношение строгого частичного порядка. Это антирефлексивность и транзитивность. На конечном множестве всегда имеется максимальный элемент. Отдельным пунктом даём свойство, что максимального элемента нет. Тогда все модели теории бесконечны. Примером будет N с отношением <.
		
		1) $(Ax)not(P(x,x)$
		
		2) $(Ax)(Ay)(Az)(P(x,y)\& P(y,z)\to P(x,z))$
		
		3) $(Ax)(Ey)P(x,y)$
		
		Искомое предложение будет конъюнкцией трёх формул из этих пунктов.
		
		\item Пусть имеется инъекция $f$ одной структуры в другую. Тогда каждому $n\ge 1$ мы ставим в соответствие $f(n)\ge 0$. При этом $f(xy)=f(x)+f(y)$ для любых $x,y\ge 1$.
		
		Полагая $x=y=1$, имеем $f(1)=0$. Ввиду инъективности, $f(x) > 0$ при $x > 1$. В частности, $k=f(2)\ge 1$ и $m=f(3)\ge 1$. Из основного условия для $f$ следует, что $f(x^n)=f(x...x)=f(x)+...+f(x)=nf(x)$ при любом $n\ge1$. В частности, $f(2^m)=mf(2)=mk$ и $f(3^k)=kf(3)=km$. Получается, что числа $2^m$ и $3^k$ заведомо не равны (первое чётно, второе нечётно), но переходят в один и тот же элемент $mk=km$, что противоречит инъективности.
		
		\item а) Заметим что они не равномощные $\Rightarrow$ нет
		
		б) $z \to -z$
		
		в) у $N$ есть наименьший, но нет наибольшего $\Rightarrow$ нет
		
		\item а) Введём обозначения для элементов первого множества. Будем считать, что действительное число $x$ из первого слагаемого R записывается как $a(x)$, а число y из второго слагаемого $R$ представляется как $b(y)$. Рассмотрим систему отрезков вида $[a(n),b(-n)]$ по всем $n\ge 1$. Она имеет пустое пересечение. В самом деле, если в пересечение попал какой-то элемент, то он имеет вид $a(x)$ или $b(y)$. В первом случае $a(n)\le a(x)$, то есть $n\le x$ для всех $n$, но так не бывает. Во втором случае $b(y)\le b(-n)$, то есть $y\le -n$ для всех $n\ge 1$, и этого также не бывает.
		
		При порядковом изоморфизме система отрезков переходит в систему отрезков, вложенная переходит во вложенную, и пересечение переходит в пересечение. Остаётся вспомнить, что в $R$ любая вложенная система отрезков имеет непустое пересечение по принципу Кантора.
		
		б) На наглядном уровне, $RQ$ есть $R$, взятое $Q$ раз, то есть мы представляем себе упорядоченное множество $Q$, и в каждой точке $q$ рассматриваем свою отдельную прямую. Это множество обладает тем свойством, что для любых $a < b$, интервал $(a,b)$ несчётен. Это верно как в случае, если $a$, $b$ принадлежат экземпляру $R$ для одного и того же рационального $q$, и тем более верно, если для разных.
		
		В $QR$ содержится экземпляр $Q$ в качестве подструктуру, а там уже интервал между любыми двумя точками счётен.
		
		\item 
		\item Для начала надо написать само алгебраическое уравнение. Оно при этом может иметь несколько корней. Их конечное число. Поэтому для корня $x$, который нас интересует, всегда есть отрезок с рациональными концами, содержащий $x$ и не содержащий других корней. Пусть его концы $a/n$ и $b/n$. Тогда добавляем через знак конъюнкции неравенства $a\le nx$ и $nx\le b$.
		
		При этом, если сигнатура поля содержит только кольцевые операции, то в их терминах надо уметь выражать предикат <=. Это несложно для случая $R$: если разность неотрицательна, то она равна квадрату какого-то числа, то есть $a\le nx$ означает, что существует $z$ такое, что $nx=a+z^2$.
		
		\item  
		
		Будем считать, что речь идёт о целых положительных числах.
		
		Равенство $a+b=c$ равносильно $ac+bc=c^2$.
		
		Рассмотрим произведение $(ac+1)(bc+1)=abc^2+(ac+bc)+1$. Если заменить $ac+bc$ на $c^2$, то получится высказывание, равносильное исходному:
		
		$(ac+1)(bc+1)=abc^2+c^2+1=(ab+1)c^2+1$
		
		Здесь каждая из частей выражена через произведение и прибавление единицы, а сама формула равносильна $a+b=c$.
		
		\item 
		\item a) Рассматривается группа рациональных чисел, не равных 0, относительно умножения. Её элементы имеют однозначное представление вида $+-p(1)^{k(1)}...p(r)^{k(r)}$, где $p(1) < ... < p(r)$ простые, и показатели целые ненулевые.
		
		Любая перестановка на множестве простых чисел задаёт автоморфизм группы. Значит, автоморфизмов не меньше, чем биекций $N$ на $N$. Их не меньше континуума, так как натуральные числа разбиваются на счётное множество пар, и в каждой паре мы элементы или переставляем, или нет. Таких биекций уже $2^N$, то есть континуум.
		
		С другой стороны, группа счётна, а отображений счётного множества в счётное не больше континуума: $N^N<=(2^N)^N~2^{NxN}~2^N$.
		
		b) Среди автоморфизмов группы есть $x\to 1/x$. Он является автоморфизмом структуры, но целые числа не сохраняет. Значит, множество $Z \not {0}$ невыразимо на языке логики предикатов.
		
		c) То же самое: автоморфизм из предыдущего пункта не сохраняет порядок <.

		\item  
		\item При естественном выборе сигнатуры, где есть кольцевые операции, достаточно заметить, что в $R$ из $x^4=1$ следует $x^2=1$, а в $C$ не следует. Поэтому формула $(Ax)(x^4=1 \to x^2=1)$ языка первого порядка будет различать эти две структуры.
		\item Обратим внимание на то, что бывают числа, у которых ровно три делителя, но не бывает множеств, у которых ровно три подмножества. Пусть Q -- двуместный предикатный символ. Формула $Q(x,y) \& Q(y,x)$ будет задавать равенство как для чисел, так и для множеств. Будем писать это условие в виде E(x,y).
		
		Нам подойдет формула, утверждающая существование такого a, для которого существуют b, c отличные от a такие, что $Q(b,a) \& Q(c,a)$, и при этом для любого t с условием Q(t,a) верна дизъюнкция E(t,a) V E(t,b) V E(t,c). На N это будет верно при a=4, а для множеств -- нет. Значит, системы не являются элементарно эквивалентными.
		
		
	\end{enumerate}
\end{document}