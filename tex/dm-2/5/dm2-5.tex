\documentclass[11pt]{article}

\makeatletter
\newcommand{\skipitems}[1]{%
	\addtocounter{\@enumctr}{#1}%
}
\makeatother

\newcommand{\numpy}{{\tt numpy}}
\usepackage{amsfonts}
\usepackage{amsmath}
\usepackage{graphics}
\usepackage{amsthm,amstext,amsfonts,bm,amssymb}
\usepackage{graphicx}
\graphicspath{ {./images/} }
\usepackage{indentfirst}
\setlength{\parindent}{5ex}
\setlength{\parskip}{1em}
\usepackage[utf8x]{inputenc} 
\usepackage[russian]{babel}
\topmargin -.5in
\textheight 9in
\oddsidemargin -.25in
\evensidemargin -.25in
\textwidth 7in


\begin{document}
	
	\author{Биктимиров Данила, группа 204}
	\title{ДЗ 5}
	\date{}
	\maketitle
	
	\medskip
	
	\begin{enumerate}
		\item Посмотрим на язык с одним двуместным предикатом $P$ и без равенства. Пусть $\phi \leftrightharpoons (\forall y)P(x,y)$ и $\psi \leftrightharpoons (\forall y)P(y,x)$. Эти формулы не эквивалентны. Тогда посмотрим на интерпретацию на области $D={1,2}$, при $P(a,b)\leftrightharpoons(a\le b)$. Тогда при $x=1$ формула $\phi$ окажется истинной, а $\psi$ ложной. С другой стороны, если добавить кванторы всеобщности по $x$, то получатся эквивалентные формулы $(\forall x)(\forall y)P(x,y)$ и $(\forall x)(\forall y)P(y,x)$. Это следует из того, что одноимённые кванторы можно переставлять, а связанные переменные переименовывать.
		\item \begin{enumerate}
			\item Является. Тогда существуют $x_0$, $y_0$ из предметной области, которые делают формулу $A(x,y,z)$ истинной при всех $z$. Тогда для любого $z$ их же далее и берём в заключении импликации: $y=y_0$, $x=x_0$.
			
			\item Является. Выведем истинность заключения из истинности посылки. Принимаем посылку. Пусть $x$ произвольно. Доказываем импликацию $A(x)\& B(x)\to C(x)$. Приняли $A(x)$ и $B(x)$. Тогда существуют $x$, для которых $A(x)$, и для которых $B(x)$. Значит, $(Ex)A(x)$ и $(Ex)B(x)$ обе верны. Из них по modus ponens верно, что $C(x)$ для всех $x$. В частности, $C(x)$ верно, ч.т.д.
			
			\item Является. Пусть посылка импликации истинна. Тогда при некотором $w=w_0$ мы имеем $(A(w)\to B(w))\to C(w)$. Истинность этой импликации означает истинность заключения или ложность посылки. В первом случае $C(w)$ истинно. Значит, истинно $(Ez)C(z)$. Это делает истинной импликацию в заключении. Теперь пусть ложна посылка первой из импликаций. Тогда $A(w)$ истинно, $B(w)$ ложно. Тем самым, истинно $(Ex)A(x)$ и ложно то, что $B(y)$ верно для всех y. Тогда у импликации из заключения оказывается ложна посылка, то есть вся эта импликация истинна.
			
			\item Не является. Построим контрпример. Возьмём $N$ в качестве предметной области, и проинтер- претируем $A(x,y,z)$ как $x=1$ или $z=1$. Тогда существуют $x=y=1$, что $A(x,y,z)$ верно при любом $z$. Также существуют $y=z=1$, что $A(x,y,z)$ верно при любом $x$. Однако не для любых $z,x$ будет верно то, что находится после кванторов в заключении импликации: взяв $z=x=2$, мы не найдём такого $y$, для которого $A(x,y,z)$.
			
			\item Не является. Рассмотрим в 3-мерном пространстве три взаимно перпендикулярные прямые, ни одна из которых не проходит через точку вида $(a,a,a)$. Например: $x=1$,$y=2$,$z$ любое; $y=1$,$z=2$,$x$ любое; $z=1$,$x=2$,$y$ любое. Предикат $A(x,y,z)$ будет означать, что точка $(x,y,z)$ принадлежит хотя бы одной из этих прямых. Посылка истинна по построению, а заключение ложно, так как точек с одинаковыми координатами мы не брали.
			
			\item Является. Если $A(x,y)$ верно не всегда, то заключение импликации ложно. Пусть $A(x,y)$ верно всегда. Тогда посылка импликации имеет вид $1\to 0$, то есть она ложна, и всё вместе истинно.
		\end{enumerate}
	\item Сначала покажем, что формула не общезначима. Пусть $N$ предметная область, и проинтер- претируем $P(x,y)$ как $x\ge y$. Посылка импликации будет истинна, так как $x\ge x$ всегда верно, и из $x\ge z$ при любом $y$ верно $x\ge y$ или $y\ge z$. В противном случае было бы $x < y < z$, что давало бы противоречие. Заключение импликации ложно, так как не существует натурального $u$ со свойством $u\ge v$ для всех $v$ (например пусть $v=u+1$). Далее покажем, что на конечной предметной области формула будет истинной. Пусть $Q$ -- отрицание предиката $P$. Тогда рефлексивность отношения $P$ даёт антирефлексивность $Q$, а из другой части условия в посылке импликации мы получаем по контрапозиции и закону де Моргана, что $Q(x,y)\& Q(y,z)$ влечёт $Q(x,z)$, то есть отношение $Q$ транзитивно. Тогда это строгий частичный порядок, и на конечном множестве для него есть максимальный элемент $u$. Его нельзя "увеличить", то есть $Q(u,v)$ всегда будет ложно, и $P(u,v)$ при любом $v$ истинно.
	\item 
	\item
	\end{enumerate}
\end{document}