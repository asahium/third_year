\documentclass[11pt]{article}

\makeatletter
\newcommand{\skipitems}[1]{%
	\addtocounter{\@enumctr}{#1}%
}
\makeatother

\newcommand{\numpy}{{\tt numpy}}
\usepackage{amsfonts}
\usepackage{amsmath}
\usepackage{graphics}
\usepackage{amsthm,amstext,amsfonts,bm,amssymb}
\usepackage{graphicx}
\graphicspath{ {./images/} }
\usepackage{indentfirst}
\setlength{\parindent}{5ex}
\setlength{\parskip}{1em}
\usepackage[utf8x]{inputenc} 
\usepackage[russian]{babel}
\topmargin -.5in
\textheight 9in
\oddsidemargin -.25in
\evensidemargin -.25in
\textwidth 7in


\begin{document}
	
	\author{Биктимиров Данила, группа 204}
	\title{ДЗ 3}
	\date{}
	\maketitle
	
	\medskip
	
	\begin{enumerate}
		
		\item \begin{enumerate}
			\item $\sum_{n=1}^{\infty} \left(p^{\frac{1}{n}}-\frac{q^{\frac{1}{n}} + r^{\frac{1}{n}}}{2} \right)$
			$$p^{\frac{1}{n}} = e^{\frac{1}{n}\ln p}=1+\frac{\ln p}{n} + \overline{\overline{o}} \left(\frac{1}{n^2}\right)$$
			Тогда: $$ \left(p^{\frac{1}{n}} - \frac{q^{\frac{1}{n}} + r^{\frac{1}{n}}}{2} \right) = \left(1+\frac{\ln p}{n} + \overline{\overline{o}} \left(\frac{1}{n^2}\right)\right)-\left( 1+\frac{\ln \sqrt{qr}}{n} + \overline{\overline{o}} \left(\frac{1}{n^2}\right) \right)=\frac{\ln{\frac{p}{\sqrt{qr}}}}{n}+\overline{\overline{o}}\left(\frac{1}{n^2}\right)$$
			И тогда, если $\ln \frac{p}{\sqrt{qr}}\not=0$ то ряд расходится. Иначе $p=\sqrt{qr}$. Тогда: $$-\frac{q^\frac{1}{n}-2\sqrt{qr}^{\frac{1}{n}}+r^\frac{1}{n}}{2}=-\frac{\left(q^{\frac{1}{n}}-r^{\frac{1}{n}}\right)^2}{2}$$
			И снова разложим: $$-\frac{\left(\left(1+\frac{\ln q}{n} + \overline{\overline{o}} \left(\frac{1}{n^2}\right)\right)-\left(1+\frac{\ln r}{n} + \overline{\overline{o}} \left(\frac{1}{n^2}\right)\right)\right)^2}{2}=-\frac{\left(\ln q - \ln r\right)^2}{8n^2}$$
			Ну а это очевидно сходится, как $\frac{1}{n^2}$.
			
			\item
		\end{enumerate}
		\item Смотрим на первые $n(p+q)$ членов: $$ \sum_{i=1}^{np} \frac{1}{2i-1} - \sum_{j=1}^{nq} \frac{1}{2j} =\left(\sum_{i=1}^{2np} \frac{1}{i} - \frac{1}{2} \sum_{j=1}^{np} \frac{1}{k}\right) - \frac{1}{2} \sum_{i=1}^{nq} \frac{1}{i} =$$ $$ =\ln (2np) + C +\overline{o}(1) - \frac{1}{2} \ln (np) - \frac{1}{2}C-\overline{o}(1)-\frac{1}{2}\ln (nq) -  \frac{1}{2} C -\overline{o}(1) =$$ $$ = \ln \left(\frac{2np}{n\sqrt{pq}}\right) + \overline{o}(1) = \ln 2 + \frac{1}{2}\ln\frac{p}{q} + \overline{o}(1)$$
		Осталось разобрались со случаем где мы берем не первые $n(p+q)$ членов.
		
		Ограничим ее ближайшими $n(p+q)$ суммами. И все внутри будет колыхаться на $\overline{\overline{o}}(1)$, ведь к этому стремится разность при $n\to \infty$. И значит все стремится к $\ln 2 + \ln \frac{p}{q}$.
		
		\item
		
		\item
		
	\end{enumerate}
\end{document}