\documentclass[11pt]{article}

\makeatletter
\newcommand{\skipitems}[1]{%
	\addtocounter{\@enumctr}{#1}%
}
\makeatother

\newcommand{\numpy}{{\tt numpy}}
\usepackage{amsfonts}
\usepackage{amsmath}
\usepackage{graphics}
\usepackage{amsthm,amstext,amsfonts,bm,amssymb}
\usepackage{graphicx}
\graphicspath{ {./images/} }
\usepackage{indentfirst}
\setlength{\parindent}{5ex}
\setlength{\parskip}{1em}
\usepackage[utf8x]{inputenc} 
\usepackage[russian]{babel}
\topmargin -.5in
\textheight 9in
\oddsidemargin -.25in
\evensidemargin -.25in
\textwidth 7in
\DeclareMathOperator*\uplim{\overline{lim}}

\begin{document}
	
	\author{Биктимиров Данила, группа 204}
	\title{ДЗ 1}
	\date{}
	\maketitle
	
	\medskip
	
	\begin{enumerate}
		
		\item $\sum^{\infty}_{k=0}q^k \cos{kx}$
		
		Пусть $z=q(\cos x + i \sin x)$
		
		Зная $z$, найдем $z^k$, но он в свою очередь равен $z^k=q^k(\cos kx + i \sin kx)$, а значит $q^k \cos kx = Re(z^k)$
		
		$$Re(\sum_{\infty}^{k=0}z^k)=\sum_{\infty}^{k=0}q^k\cos kx$$
		$$\sum_{\infty}^{k=0}z^k=\frac{1}{1-z}$$
		Тогда получаем ответ $Re(\frac{1}{1-q(\cos x+i\sin x)})$
		\item
		\begin{enumerate}
			\item Заметим, что $a^2 + b^2 \ge 2|ab|$. Так как $\sum_{n=1}^{\infty}a_n^2$ и  $\sum_{n=1}^{\infty}a_n^2$ сходятся, тогда сходится и $\sum_{n=1}^{\infty}a_n^2 + \sum_{n=1}^{\infty}b_n^2 = \sum_{n=1}^{\infty}(a_n^2+b_n^2)$ и тогда из признака сранения сходится и $\sum_{n=1}^{\infty}|a_nb_n|$
			\item Воспользуемся первым пунктом и подставим вместе $b_n=\frac{1}{n}$, тогда получаем что $\sum_{n=1}^{\infty} \frac {a_n}{n}$ сходится
		\end{enumerate}
		\item Рассмотрим гармонический ряд $\sum_{n=1}^{\infty}\frac{1}{n}$. $\forall p\in \mathbf{N} \lim_{n\to \infty}\sum_{k+n}^{n+p}a_k=0$. 
		$$0\le \lim_{n\to \infty}\sum_{k=n}^{n+p}\frac{1}{k} \le \lim_{n\to \infty}\frac{p}{n}=0$$
		ТО есть для гармонического ряда условие выполняется, но он как мы занем расходится. То есть ответ нет.
		\item Пусть $\epsilon$-произвольное положительное число, а $p$-произвольное натуральное число, тогда $$|\frac{cos(2^{n+1})}{(n+1)^2}+\frac{cos(2^{n+2})}{(n+2)^2}+...+\frac{cos(2^{n+p})}{(n+p)^2}|\le|\frac{cos(2^{n+1})}{(n+1)^2}|+|\frac{cos(2^{n+2})}{(n+2)^2}|+...+|\frac{cos(2^{n+p})}{(n+p)^2}|\le$$
		$$\le\frac{1}{(n+1)^2}+\frac{1}{(n+2)^2}+...+\frac{1}{(n+p)^2}< \frac{1}{n(n+1)}+\frac{1}{(n+1)(n+2)}+...+\frac{1}{(n+p-1)(n+p)}=\frac{1}{n}-\frac{1}{n+p}<\frac{1}{n}<\varepsilon$$
		откуда $n>\frac{1}{\varepsilon}, N(\varepsilon)=[\frac{1}{\varepsilon}]$, значит ряд сходится
		\item
		\begin{enumerate}
			\item 
			
			\item $\sum_{n=2}^{\infty} (\frac{n-1}{n+1})^{n(n-1)}$
			
			Применим радикальный признак Коши:
			
			$$ 
			\uplim_{n \to +\infty} \sqrt[^n]{\left(\frac{n-1}{n+1} \right) ^{n(n-1)}} =
			\uplim_{n \to +\infty} \left( \frac{n-1}{n+1} \right)^{n-1} = 
			\uplim_{n \to +\infty} \left( 1 - \frac{2}{n} \right)^n =
			e^{-2} < 1 \Rightarrow \text{ ряд сходится.}$$
			\item Пусть $$c_1=\sqrt{2},\quad c_2=\sqrt{2+\sqrt{2}},\quad c_{n+1}=\sqrt{2+c_n}$$
			и $d_n=\frac{c_n}{2}$, тогда $d_1=\cos{\frac{\pi}{4}}$ и $d_{n+1}=\sqrt{\frac{1+d_n}{2}}$, также заметим, что $c_n=2\cos{\frac{\pi}{2^{n+1}}}$, тогда
			$$\sqrt{2} = 2\sin\frac{\pi}{4},\quad \sqrt{2-\sqrt{2}}=2\sin\frac{\pi}{8},\qquad \sqrt{2-c_n} =  2\sin\frac{\pi}{2^{n+2}}$$
			т.е. мы хотим узнать, сходится ли $\sum_{n\geq 0}2\sin\frac{\pi}{2^{n+2}}$, а она сходится так как $0<\sin\frac{\pi}{2^{n+2}}<\frac{\pi}{2^{n+2}}$
		\end{enumerate}
		
		
	\end{enumerate}
\end{document}