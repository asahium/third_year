\documentclass[11pt]{article}

\makeatletter
\newcommand{\skipitems}[1]{%
	\addtocounter{\@enumctr}{#1}%
}
\makeatother

\newcommand{\numpy}{{\tt numpy}}
\usepackage{amsfonts}
\usepackage{amsmath}
\usepackage{graphics}
\usepackage{amsthm,amstext,amsfonts,bm,amssymb}
\usepackage{graphicx}
\graphicspath{ {./images/} }
\usepackage{indentfirst}
\setlength{\parindent}{5ex}
\setlength{\parskip}{1em}
\usepackage[utf8x]{inputenc} 
\usepackage[russian]{babel}
\topmargin -.5in
\textheight 9in
\oddsidemargin -.25in
\evensidemargin -.25in
\textwidth 7in


\begin{document}
	
	\author{Биктимиров Данила, группа 204}
	\title{ДЗ 2}
	\date{}
	\maketitle
	
	\medskip
	
	\begin{enumerate}
		
		\item Чтобы никто не принес свой ноут домой надо, чтоб каждый кто взял свой его потерял. Пусть $m$ человек взяли свой ноут. Это 
		$$C^m_n \cdot !(n-m) = C^m_n \cdot \left(\sum_{k=0}^{n-m}\frac{(-1)^k}{k!}\right) \cdot (n-m)!$$
		
		Ну и чтоб все $m$ потеряли свой ноут нужна "удача" в $p^m$. Так как общее число исходов $n!$, то получаем:
		$$\frac{\sum_{m=0}^{n}\left(C^m_n\cdot \left( \sum_{k=0}^{n-m} \frac{(-1)^k}{k!} \right) \cdot (n-m)!\cdot p^m \right)}{n!}$$
		
		\item
		$$P(A, \text{отр.})=(0.9 \cdot 0.05 + 0.1 \cdot 1)=0.145\:\:\: P(A, \text{пол.})=0.855$$
		$$P(B, \text{отр.})=(0.5 \cdot 0.05 + 0.5 \cdot 1)=0.525\:\:\: P(A, \text{пол.})=0.475$$
		$$P(C, \text{отр.})=(0.2 \cdot 0.05 + 0.8 \cdot 1)=0.81\:\:\: P(A, \text{пол.})=0.19$$
		$$P(D, \text{отр.})=1\:\:\: P(A, \text{пол.})=0$$
		$P(X,Y)=P(X,\text{отр.})\cdot P(Y,\text{отр.})\cdot P(Y,\text{пол.})$, тогда
		$$P(A,B)=0.145^2\cdot 0.475\:\:\: P(A,C)=0.145^2\cdot 0.19$$
		$$P(B,A)=0.525^2\cdot 0.855\:\:\: P(B,C)=0.525^2\cdot 0.19$$
		$$P(C,A)=0.81^2\cdot 0.855\:\:\: P(C,B)=0.81^2\cdot 0.475$$
		$$P(D,A)=1^2\cdot 0.855\:\:\: P(D,B)=1^2\cdot 0.475\:\:\: P(D,C)=1\cdot 0.19$$
		Тогда с учетом того, что все исходы равновероятны $P(\text{отр.},\text{отр.},\text{пол.})=\frac{1}{12}\cdot \sum_{XY} P(X,Y)=\frac{2.69462275}{12}$, тогда
		$$P(A\:\text{второй}|\text{отр.},\text{отр.},\text{пол.})=\frac{P(B,A) + P(C,A)+P(D,A)}{12P}=0.61...$$
		$$P(B\:\text{второй}|\text{отр.},\text{отр.},\text{пол.})=\frac{P(A,B) + P(C,B)+P(D,B)}{12P}=0.61...$$
		$$P(C\:\text{второй}|\text{отр.},\text{отр.},\text{пол.})=\frac{P(B,C) + P(A,C)+P(D,C)}{12P}=0.61...$$
		$$P(D\:\text{второй}|\text{отр.},\text{отр.},\text{пол.})=\frac{P(B,D) + P(C,D)+P(A,D)}{12P}=0.61...$$
		
	\end{enumerate}
\end{document}