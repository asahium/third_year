\documentclass[11pt]{article}

\makeatletter
\newcommand{\skipitems}[1]{%
	\addtocounter{\@enumctr}{#1}%
}
\makeatother

\newcommand{\numpy}{{\tt numpy}}
\usepackage{amsfonts}
\usepackage{amsmath}
\usepackage{graphics}
\usepackage{amsthm,amstext,amsfonts,bm,amssymb}
\usepackage{graphicx}
\graphicspath{ {./images/} }
\usepackage{indentfirst}
\setlength{\parindent}{5ex}
\setlength{\parskip}{1em}
\usepackage[utf8x]{inputenc} 
\usepackage[russian]{babel}
\topmargin -.5in
\textheight 9in
\oddsidemargin -.25in
\evensidemargin -.25in
\textwidth 7in


\begin{document}
	
	\author{Биктимиров Данила, группа 204}
	\title{ДЗ 10}
	\date{}
	\maketitle
	
	\medskip
	
	\begin{enumerate}
		
		\item Рассмотрим с.в. $W=X⋅Y$  её ф.р. $FW(\alpha)$ равна нулю при $\alpha<0$ и единице при $\alpha>1$. При $\alpha \in [0;1]$ по методу монте-карло получим $F_{W}(\alpha)=\iint\limits_{[0;1]^2 \cap xy < \alpha}1\,dx\,dy = \alpha+\int\limits_{\alpha}^{1}\frac{\alpha}{x}\,dx = \alpha - \alpha\cdot\ln \alpha$
		
		Тогда плотность этой с.в. равна $f_{W}(\alpha)=-\ln \alpha$. Дальше рассмотри с.в $S=WZ$, у неё так же с.в. $FS(\beta)$ равна нулю при $\beta<0$ и единице при $\beta>1$ и снова по методу монте-карло получаем, что при $\beta\in [0;1]$
		$F_{S}(\beta)=\iint\limits_{[0;1]^2 \cap s^z < \beta}(-\ln s)\,ds\, dz$
		
		Если сделать замену $u=\ln s$,$v=z$, то $(s;z)\in [0;1]^2\to (u;v)\in (−\infty;\ln\beta]×[0;1]$ получим
		$$F_{S}(\beta)=\iint\limits_{(-\infty;\ln\beta]\times [0;1] \cap uv < \ln\beta}(-u\cdot e^u)\,du\, dv =\int\limits_{-\infty}^{\ln\beta}(-u\cdot e^u)\,du \int\limits_{\frac{\ln\beta}{u}}^{1} 1\;dv = \ldots = \beta$$
	\end{enumerate}
\end{document}